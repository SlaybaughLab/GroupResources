% NEUTRONICS BEAMER TEMPLATE -- START EDITING IN LINE 90

\documentclass[xcolor=x11names, compress]{beamer}

\definecolor{CoolBlack}{rgb}{0.0, 0.18, 0.39}
\definecolor{byellow}{rgb}{0.55037, 0.38821, 0.06142}

\usepackage[T1]{fontenc}
\usepackage[utf8]{inputenc}
\usepackage{lmodern}
\usepackage{amsthm}
\usepackage{amssymb}
\usepackage{amstext}
\usepackage{bm}
\usepackage{graphicx}
\usepackage{epstopdf}
\usepackage{amsmath}
\usepackage{setspace}
\usepackage{tikz}
\usepackage{Tabbing}
\usepackage{mathrsfs}
\usepackage[mathcal]{euscript}
\usepackage{epsfig}
\usepackage{changepage}
\usepackage{xcolor}
\usepackage{fancyvrb}
\usepackage{caption}
\usepackage{color}
\usepackage[version=3]{mhchem}
\usepackage{hyperref}
\usepackage{multirow}
\usepackage[firstpage]{draftwatermark}
\usepackage{animate}
\usepackage{appendixnumberbeamer}



\usetikzlibrary{decorations.fractals}

\setbeamerfont{title like}{shape=\scshape}
\setbeamerfont{frametitle}{shape=\scshape}

\setbeamercolor*{lower separation line head}{bg=CoolBlack}
\setbeamercolor*{normal text}{fg=black,bg=white}
\setbeamercolor*{alerted text}{fg=dgreen} 
\setbeamercolor*{example text}{fg=black}
\setbeamercolor*{structure}{fg=black}

% Margins
\mode<presentation>
{
  \definecolor{berkeleyblue}{HTML}{003262}
  \definecolor{berkeleygold}{HTML}{FDB515}
  \usetheme{Boadilla}      % or try Darmstadt, Madrid, Warsaw, Boadilla...
  %\usecolortheme{dove} % or try albatross, beaver, crane, ...
  \setbeamercolor{structure}{fg=berkeleyblue,bg=berkeleygold}
  \setbeamercolor{palette primary}{fg=berkeleyblue,bg=berkeleygold}
  \setbeamercolor{palette secondary}{fg=berkeleyblue,bg=berkeleygold}
  \setbeamercolor{palette tertiary}{bg=berkeleyblue,fg=white}
  \usefonttheme{structurebold}  % or try serif, structurebold, ...
  \useinnertheme{circles}
  \setbeamertemplate{caption}[numbered]
  \usebackgroundtemplate{}
}

%% Beamer Layout %%%%%%%%%%%%%%%%%%%%%%%%%%%%%
\useoutertheme[subsection=false,shadow]{miniframes}
%\useinnertheme{default}
%\usefonttheme{serif}
%\usepackage{palatino}
%\usepackage{tabu}
% addition of color
\definecolor{dgreen}{rgb}{0.,0.6,0.}
\definecolor{RawSienna}{cmyk}{0,0.72,1,0.45}
%\usepackage[sorting=none]{biblatex}
\mode<presentation>

% Links
\definecolor{links}{HTML}{003262}
\hypersetup{colorlinks,linkcolor=,urlcolor=links}

% columns
\renewcommand{\(}{\begin{columns}}
\renewcommand{\)}{\end{columns}}
\newcommand{\<}[1]{\begin{column}{#1}}
\renewcommand{\>}{\end{column}}

%%%%%%%%%%%%%%%%%%%%%%%%%%%%%%%%%%%%%%%%%%%%%%%%%%%%%%%%%%%%%%%%%%%%%%%%%%%%%%%%%%%%%%%%%%%%%%%%%%%%%%%%%%%%%%%%%%%%%%%%%%%%%%%%%%%%%%%%%%%%%%%%%%%%%%%%%

% START EDITING HERE

%THIS IS THE BEAR WATERMARK IN THE COVER SLIDE; YOU MAY EXCLUDE IT
\SetWatermarkText{\includegraphics{0calbear.jpg}}
\SetWatermarkAngle{0}
\SetWatermarkScale{0.61}
%%%%%%%%%%%%%%%%%%%%%%

%THIS IS THE INFO AT THE BOTTOM OF YOUR FRAMES
\title{Deadly Virus and Outbreaks}
\author{Your Name and Co-Author's}
\date{October 1$^{st}$, 1998}

%%%%%%%%%%%%%%%%%%%%%%%%%%%%%%%%%%%%%%%%%%%%%%%%%%

\begin{document}

%%%%%%%%%%%%%%%%%%%%%%%%%%%%%%%%%%%%%%%%%%%%%%%%%%%%%%
\begin{frame}[plain]
%THIS IS THE TITLE OF THE TALK
\title{HOW TO IDENTIFY A DEADLY VIRUS AND PREVENT AN OUTBREAK \\ (A Template for Beamer)}

%FEEL FREE TO EDIT THE COVER LAYOUT AS NEEDED
\author{
\begin{tabular}{ccc}
\multirow{6}{65pt}{\includegraphics[height=2.5cm]{0bk.eps}}
& &
\multirow{6}{65pt}{\includegraphics[height=2.5cm]{0neutronics.png}} \\
&{\small\textcolor{blue}{University of California, Berkeley}} &\\
&{\small\textcolor{blue}{Department of Nuclear Engineering}}&\\
&{\small \textcolor{blue}{Neutronics Research Group}}&\\
& &\\
%YOUR NAME
&\textbf{Richard Vasques}&\\
%YOUR CO-AUTHORS
&\textbf{Jill Valentine$^{1}$}&\\
%CO-AUTHOR'S INSTITUTION
\multicolumn{3}{c}{\small\textcolor{blue}{$^{1}$Raccoon City S.T.A.R.S.}}
\end{tabular}}
\date{\vspace{-20pt}\\ October 1$^{st}$, 1998 \\ Raccoon City \\ Umbrella Corporation}
\titlepage
\end{frame}

% --------------------------------------------------------------
% THIS IS THE OUTLINE OF YOUR TALK -- YOU MAY SKIP THIS FRAME
\begin{frame}[fragile]{Outline}
	\begin{itemize}
	\item{Identifying the infected}
	\item{Killing zombies}
	\begin{itemize}
	    \item{Locating the target}
	    \item{Performing headshots}
	\end{itemize}
	\item{Treating bites}
	\item{Final solution}
	\item{Future work}
	\end{itemize}
\end{frame}


%----------------------------------------------------------%
% WHEN YOU START A NEW SECTION, IT WILL SHOW UP AS A CLICKABLE SHORTCUT AT THE TOP OF YOUR FRAMES
\section{\scshape Identifying Infected}

%%%%%%%%%%%%%%%%%%%%%%%%%%%%%%%%%%%%%%%%%%%%%%%THIS IS A FULL FRAME
\frame[c]{
	\frametitle{Identifying the infected (basic frame, box, math, etc.)}
There is one good rule of thumb to identify the infected:
\vspace{10pt}

%YOU CAN HAVE BLOCKS FOR EMPHASIS
\begin{block}{Is he or she trying to eat you?}
If the answer is yes, there is a good chance he or she is infected.
\end{block}
\vspace{10pt}

%AND AS MANY EQUATIONS AS YOU WANT
Here is a nice mathematical model:
\[
P_i = 1- \frac{R}{B}
\]
where $P_i$ is probability of infection, $R$ is numer of limbs that are rotting or missing, and $B$ is total limbs in the body. 
}
%%%%%%%%%%%%%%%%%%%%%%%%%%%%%%%%%%%%%%%%%%%%%%

\section{\scshape Killing Zombies}

\frame[c]{
	\frametitle{Locating the target (using the animate package)}
This part is easy; they will llikely come to you.		
\vspace{5pt}

%YOU CAN MAKE ANIMATIONS WITH THE ANIMATE PACKAGE. THIS ONE IS IN A CONSTANT LOOP
	\begin{center}
\animategraphics[scale=.5,autoplay,loop]{15}{H}{1}{30}
\end{center}
}

\frame[c]{
	\frametitle{Performing headshots (using the animate package)}
Aim at the head and shoot. 
%YOU CAN ALSO USE BUTTONS. THIS IS USEFUL BECAUSE IT ALLOWS YOU TO GO FRAME BY FRAME, SKIP TO THE END, INCREASE/DECREASE SPEED OF ANIMATION, ETC. GOOD FOR PRESENTING RESULTS, PLOTS, ETC.	
	\begin{center}
	\animategraphics[scale=.6,controls,loop]{10}{Z}{1}{15}
\end{center}
}

\section{\scshape Treating Bites}

\frame[c]{
	\frametitle{Try not to be bitten (using columns, including pictures)}
	
There isn't really a treatment. If you do get bitten...
\vspace{5pt}

%YOU CAN USE COLUMNS TO MAKE YOUR SLIDE MORE ORGANIZED/APPEALING
\begin{columns}
	\column{0.25\linewidth}
	%YOU MAY INCLUDE GRAPHICS/PICTURES/PLOTS DIRECTLY IN THE TEXT
	\includegraphics[scale=.25]{zombie.jpg}
	\column{0.65\linewidth}
\begin{itemize}
\item it will hurt.	
\item like, REALLY hurt.
\item you might become food.
\item you may be able to escape, and then...
\begin{itemize}
\item ... your vocabulary will be reduced.
\item ... you will start to smell bad.
\item ... people will start looking delicious.
\end{itemize}  
\end{itemize}
\end{columns}
\vspace{5pt}

Summarizing: try not to get bitten.
}

\section{\scshape Final Solution}

\frame[c]{
	\frametitle{Final solution (``pause'' command)}
	%PRESENT YOUR CONCLUSIONS
Umbrella is deploying a missile.
\vspace{10pt}

%THE PAUSE COMMAND CREATES A PAUSE IN YOUR SLIDE AUTOMATICALLY, BY DUPLICATING THE SLIDE IN THE PDF FILE 
\pause
Seriously. The whole city is going ``boom'' within a few minutes. 
}

\section{\scshape Future Work}
\frame[c]{
	\frametitle{Future work (add future work)}
%DISCUSS FUTURE WORK
Sort of irrelevant, seeing we are about to explode. You should keep your questions short.
}



%%%%%%%%%%%%%%%%%%%%%%%%%%%%%%%%%%%%%%%%%%%%%%%%%%%%%%
% ANY SLIDES AFTER THE \appendix COMMAND WILL NOT BE COUNTED IN THE DOCUMENT, BUT WILL STILL BE THERE
\appendix
\frame[c]{
	\frametitle{Appendix Frame}

This is an appendix slide. It's not counted with the regular talk; the frame count restarts. 
\vspace{10pt}

Check some example beamer talks in my website: \href{https://ricvasques.github.io/pages/talks.html}{\textcolor{blue}{ricvasques.github.io}}.
\vspace{10pt}

Thanks for playing!
\vspace{40pt}

\hfill\textcolor{teal}{Richard Vasques, April, 2016}
}


\end{document}
